\documentclass{article}
\usepackage[a4paper, total={16cm, 27cm}]{geometry}
\usepackage{polski}

\usepackage{float}
\usepackage{amssymb}
\usepackage{wasysym}
\usepackage[colorlinks=true, linkcolor=black, urlcolor=blue, pdftitle={Overleaf Example}]{hyperref}
\usepackage{todonotes}
%\usepackage[disable]{todonotes}

\author{Jacek Winiarczyk}
\date{}
\title{\Huge\textbf{Czy może istnieć nic?}}
%start: 3 kw 2022 r.

\begin{document}
\maketitle

\section{Wstęp}
Czym jest nic? Wyobrażenie sobie obiektu wykonanego z niczego nie jest łatwe.
Żeby wiedzieć gdzie jest \textbf{\textit{nic}} trzeba mieć wokół \textbf{\textit{coś}} dla porównania
\footnote{Powrót do filozoficznego pytania „Dlaczego istnieje raczej coś niż nic?”}.
Stworzenie takiego obiektu wydaje się natomiast łatwe~-~należy tylko zabrać materię z jego przestrzeni i tyle.
Nic twórczego, a jednak wykonanie tego jest problematyczne.
%///////////////////////////////////////////////////////////////////////////////////

\section{Problemy z \textit{niczym} w materialnym pojemniku}
Stworzenie pustej przestrzeni, bez cząsteczek, czyli klasycznej \textbf{\textit{próżni}},
wydaje się być banalne, należy wziąć wytrzymałą butlę,
pompę i odessać powietrze. Jest próżnia - temat zamknięty?

No nie. W normalnych warunkach (na poziomie morza) na ścianki butli naciska powietrze o sile 101~325N na metr kwadratowy.
Gdyby zabrać powietrze z butli na ścianki od wewnątrz nie będzie działać żadna siła i
cząsteczki butli będące w ciągłym ruchu zaczną się od butli odrywać i wypełniać w niej przestrzeń.

Można pomyśleć że schłodzenie butli zmniejszy ruch cząstek i zaprzestanie ich odrywaniu.
Jest to jednak niemożliwe ponieważ zgodnie z 3. zasadą termodynamiki otrzymanie temperatury 0~K
(braku jakiegokolwiek ruchu) jest niewykonalne w skończonej liczbie kroków
\footnote{$T\to 0$~K~$\Longleftrightarrow\Delta S \to 0\Longleftrightarrow W \to \infty$},
więc zawsze będą istnieć odrywające się cząsteczki.
%///////////////////////////////////////////////////////////////////////////////////

\section{Problemy ze skonstruowaniem próżni w Kosmosie}
\todo[inline]{zanieczyszczenie orbity Ziemi, pył kosmiczny}
Do każdego miejsca we wszechświecie dociera światło gwiazd, a jest ono energią. $E=hf=mc^2$.
Powstaje pytanie czy ze słońca będącego na jednym końcu wszechśwta do obserwatora będącego na drugim dotarłaby wystarczająca ilość energii by zamienić się w masę i zaburzyć próżnię?

Mierząc światło słoneczne spektrometrem można się dowiedzieć o długości fali i jej intensywności.

\begin{figure}[H]
	\centering
	\caption{Zależność natężenia promieniowania od długośći fali światła słonecznego}
	\includegraphics[scale=0.5]{spectrometer.jpg}
	%{\tiny Żródło:~\url{https://www.researchgate.net/figure/Suns-spectrum-measured-through-a-spectrometer-The-spectrum-is-not-calibrated-in_fig4_325796516}}
	{\tiny Żródło:~\url{https://fondriest.com/environmental-measurements/parameters/weather/photosynthetically-active-radiation/}}
\end{figure}

Najintensywniejszym kolorem śwtała słonecznego jest niebieskozielony o długości $\lambda \approx 500$nm.
Podstawiając tę informację do wzoru Wiena otrzymuje się temperaturę fotosfery $T=\frac{b}{\lambda}\approx 5~796$~K.
Ze wzoru Stefana-Boltzmanna wychodzi moc słońca na każdy jego metr kwadratowy $\Phi_{wyslane}=\sigma T^4\approx 63,9 \cdot 10^{6}\frac{W}{m^2}$.
Przyjmując promień słońca za 696 000km otrzymuje się pole powierzchni A=$4\pi R^2\approx 6,087\cdot 10^{18}~m^2$.

\todo[inline]{Poniższe wyliczenie są błędne}
Z powyższych danych można policzyć z jaką mocą słońce promieniuje we wszystkich kierunkach $P=\Phi_{wyslane} A\approx 1,945\cdot 10^{29}$~W.
Następnie należy policzyć ile tej mocy dociera do obserwatora na drugim krańcu wszechświata ($\diameter =8,8\cdot 10^{26}$~m).
$\Phi_{odebrane}=\frac{P}{4\pi d^2}=\frac{\Phi_{wyslane} 4\pi R^2}{4\pi d^2}=\frac{\Phi_{wyslane} R^2}{d^2}=3,997\cdot 10^{-17} \frac{W}{m^2}$
Przyjmując że obserwator ma 1 $m^2$ to w ciągu jednej sekundy dociera energia $E=\Phi_{odebrane}\cdot 1 m^2 \cdot 1s=3,997\cdot 10^{-17} J$
%Ta jest w stanie zamienić się w masę $m=\frac{E}{c^2}=4,447\cdot10^{-34}$ kg.
\todo[inline]{dopracuj, co z ubytkiem po drodze? Paradoks Olbersa}

Oprócz energii Słońce (jak każda gwiazda) emituje neutrina. Jest to spowodowane cyklem protonowym,
dzięki któremu gwiazda zamienia wodór w hel i energię.
\todo[inline]{O neutrinach, bombardowanie, może o procesach w gwiazdach, ile np. słońce emituje neutrin}

Tak też żeby w kosmosie było nic, wszędzie musi być nic, ponieważ gwiazda będąca na drugim końcu wszechświata jest w stanie zabużyć próżnię.
%///////////////////////////////////////////////////////////////////////////////////

\section{Czy \textit{nic} jest względne?}
\todo[inline]{O efekcie Unruha - dla stacjonarnego obserwatora w X może nie być cząsteczek, a dla poruszającego się z $v\approx c$ są}
%///////////////////////////////////////////////////////////////////////////////////

\section{Czy \textit{nic} może wogóle istnieć?}
\todo[inline]{o kwantowej próżni, o cząsteczkach wirtualnych, kreacji par, fale grawitacyjne}
%///////////////////////////////////////////////////////////////////////////////////

\section{Czy możemy zmierzyć \textit{nicość} próżni?}
Z zasady nieoznaczoności heisenberga:
$$ \Delta x \Delta p_x \geq \frac{\hbar}{2} $$
$$\therefore \Delta E \Delta t \geq \frac{\hbar}{2}$$
$$\therefore \Delta mc^2 \Delta t \geq \frac{\hbar}{2}$$
$$\therefore \Delta m \geq \frac{\hbar}{2c^2\Delta t}$$
Oznacza to że możliwe jest określenie masy w obrębie próżni z niepewnością $\geq \frac{\hbar}{2c^2\Delta t}$\\
Na przykład:

Jedną z najmniejszych mas ma neutrino\footnote{Początkowo uważano że nie ma masy}
$m=0,04 eV=7,13\cdot10^{-38}$kg.
Gdy przyjmie się niepewność pomiaru masy rzędu masy neutrina,
wtedy niepewność pomiaru czasu jest
$\geq~\frac{\hbar}{2c^2\Delta m_{neutrino}}=8,228\cdot10^{-15}$s.
Dla porównania czas połowicznego rozpadu $^{8}$Be~wynosi $81.9\cdot 10^{-18}$ s.

Gdyby chcieć mierzyć czas z jedną z największych dokładności na przykład czasem Plancka\\
$\Delta t_{min}=t_{Plancka}=\frac{l_{Plancka}}{c}=5,391\cdot10^{-44}$, to masę można określić z niepewnością
$ \geq \frac{\hbar}{2c^2t_{Plancka}} = 1,088\cdot 10^{-8}kg$,
dla porównania masa cząsteczki węgla wynosi $1,99\cdot10^{-26}$kg

Powyższe przykłady pokazują że nie możliwe jest stwierdzenie czy próżnia jest napewno pusta.
%///////////////////////////////////////////////////////////////////////////////////

\section{Podsumowanie}
Wszystko ze wszystkim oddziałuje oraz nie istnieje ciało o $T=0~K$, przez to bezruch jest niemożliwy,
a nawet gdyby go uzyskać przyroda nie pozwoli mu istnieć przez m.in. oddziaływanie grawitacyjne.
A skoro bezruch jest niemożliwy to utworzenie a tym bardziej utrzymanie próżni idealnej jest niewykonalne,
nie wspominając o kreacji par i ciągłym bombardowaniu neutrinami które nie ułatwiają zadania.
Pomijając to wszystko,nawet gdyby mieć próżnie to nie da się być w pełni pewnym że nic w niej nie ma.
%///////////////////////////////////////////////////////////////////////////////////

\section{Bibliografia}
Witold Mizerski „Tablice Fizyczno-Astronomiczne”\\
Lewrence M. Krauss „Wszechświat z niczego”\\
Andrzej Dragan „Kwantechizm”

\tableofcontents
\listoftodos

\end{document}